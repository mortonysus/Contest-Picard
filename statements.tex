\documentclass{article}
\usepackage[utf8]{inputenc}
\usepackage{geometry}
 \geometry{
 a4paper,
 total={170mm,257mm},
 left=20mm,
 top=20mm,
 }
 \usepackage{graphicx}
 \usepackage{titling}

 \title{Pikard}
\author{Egor Zubkov, Ivan Palchenkov}
\date{January 2023}
 
 \usepackage{fancyhdr}
\fancypagestyle{plain}{%  the preset of fancyhdr 
    \fancyhf{} % clear all header and footer fields
    \fancyfoot[L]{\thedate}
    \fancyhead[L]{Pikard problem}
    \fancyhead[R]{\theauthor}
}
\makeatletter
\def\@maketitle{%
  \newpage
  \null
  \vskip 1em%
  \begin{center}%
  \let \footnote \thanks
    {\LARGE \@title \par}%
    \vskip 1em%
    %{\large \@date}%
  \end{center}%
  \par
  \vskip 1em}
\makeatother

\usepackage{lipsum}  
\usepackage{cmbright}

\begin{document}

\maketitle

\noindent\begin{tabular}{@{}ll}
    Team: \theauthor\\
    Time limit: X seconds\\
    Memory limit: 256mb
\end{tabular}

\section*{Problem}
You are given a first-order differential equation of the form $y' = f(t,y)$. Your task is to find the value of its kth approximation $y_k$ at the specified point

\section*{Input}
The first line of the input contains a differential equation in the format $y' = f(t,y)$. The equation can contain $sin, cos, exp$ and degree functions (sympy format).\\\\
The second line contains an integer $1 <= k <= 10$ - the iteration number for which you need to find the approximation value at the point.\\\\
The third line contains two computable expressions $t_0$ and $y_0$ (sympy format) - the initial condition for solving the Cauchy problem. \\\\
The fourth line contains a computable expression $t_k$ (sympy format) - the point where you need to find the value of the kth approximation. \\\\
It is guaranteed that it is possible to find Picard approximations for a given differential equation in elementary functions.

\section*{Output}
The only real number $x = y_k(t_k), -10^{12} <= x <= 10^{12}$. The answer will be correct with an error of no more than 8 decimal places
\section*{Samples}
\begin{tabular}{p{8cm}||p{8cm}}
 \hline
 picard.in & picard.out\\
 \hline\hline
 $y' = t**3 + 5*y*sin(2*t)$ & 7125.421055131611 \\
 5 \\
 pi 1 \\
 3*pi \\
 \hline
 $y' = t*y*(216*t**2 - 1) + 3*exp(8*t)$ & -694570194783.7665 \\
 3 \\
 2 5 \\
 1 \\
 \hline
\end{tabular}

\end{document}
